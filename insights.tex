\documentclass[a4paper,12pt]{article}
\usepackage{setspace}
\usepackage{graphicx}
\usepackage{natbib}
\bibpunct{(}{)}{;}{a}{}{,}
\usepackage[english]{babel}
\usepackage[utf8]{inputenc}
\usepackage {enumerate}
\parindent=1.25cm

\newcommand{\farcm}{\mbox{\ensuremath{.\mkern-4mu^\prime}}}%
\newcommand{\farcs}{\mbox{\ensuremath{.\!\!^{\prime\prime}}}}
\newcommand{\ii }{\'{\i}}
\newcommand{\cc }{\c c}
\newcommand{\CC }{\C C}
\newcommand{\cca}{\c ca }
\newcommand{\CCA}{\C CA }
\newcommand{\ao}{\~ao }
\newcommand{\AO}{\~AO }
\newcommand{\cao}{\c c\~ao }
\newcommand{\CAO}{\C C\~AO }
\newcommand{\oes}{\~oes }
\newcommand{\OES}{\~OES }
\newcommand{\coes}{\c c\~oes }
\newcommand{\COES}{\C C\~OES }
\newcommand{\eq}{\begin{equation}}
\newcommand{\feq}{\end{equation}}
\newcommand{\dm}{\begin{displaymath}}
\newcommand{\fdm}{\end{displaymath}}
\newcommand{\eqn}{\begin{eqnarray}}
\newcommand{\feqn}{\end{eqnarray}}
\newcommand{\grau}{^{\circ}}
\newcommand{\ba}{\arrowvert_{t_1}^{t_2}}
\newcommand{\bc}{\arrowvert_{0^{\circ} {\rm C}}^{t_2}}
\newcommand{\bb}{\arrowvert_{0^{\circ} {\rm C}}^{t_1}}
\newcommand{\Ms}{$\mathrm{M}_{\odot}$}
%\pagestyle{headings}
\usepackage{amssymb}
\newcommand{\reg}[1]{#1$^{\tiny{\circledR}}$}

%\topmargin -2cm
\topmargin -0.7cm   %inferior 1.7 superior -4
%\oddsidemargin -0.7cm
\oddsidemargin 0cm  %esquerda -2.5 direita 2.5
%\textwidth 16.5cm
\textwidth 17cm
%\textheight 24cm
\textheight 24.7cm

\begin{document}

%\pagestyle{headings}
{\rm
\pagebreak
%\pagestyle{empty}
\thispagestyle{empty}
{\tiny{.}}


\vspace{1.1cm}
\begin{center}
\hspace{1cm} {\textbf{\Large{Introdu\c c\~ao a An\'alise de Dados}}}


\vspace{0.5cm}
\hspace{12.5cm} Cleiton Carillo de Souza
\end{center}
\vspace{0.8cm}



%-------------------------------------------------------------------

\hspace{-0.8cm} Em virtude do aumento exponencial do consumo e gera\c c\~ao de dados em todas as \'areas do conhecimento,
em especial no ramo dos neg\'ocios, as empresas precisam encontrar meios de r\'apido processamento dos
seus dados, pensando na busca de oportunidades promissoras para o futuro da empresa, criando
hip\'oteses que possam alavancar as vendas.

\hspace{-0.8cm} Desta forma, a an\'alise de dados est\'a fundamentada na obtenção de informa\c c\~oes \'uteis a partir dos dados,
gerando, por consequ\^encia, insights para as empresas e ajudando-as na tomada de decis\~oes assertivas
e orientadas de acordo com os resultados da an\'alise.
\vspace{0.18cm}

{\center

Neste reposit\'orio, temos o sequinte desafio a ser resolvido: 

Certa empresa de bermudas possui cinco lojas e est\'a querendo aumentar as vendas. 

Desta forma, fazendo uso da base de vendas, o que ela deve fazer para

alcan\c car tal objetivo?

}
\vspace{1.3cm}


\hspace{-0.8cm} {\bf{\large{Insights}}}
\vspace{0.5cm}

\hspace{-0.8cm} Ao analisar os dados da base de vendas, notamos que a loja situada no Iguatemi Campinas possui um faturamente bem acima das demais, com um valor final R$\$$ 41.720,00. Dando uma \^enfase maior a esta loja, percebemos que um dos produtos est\'a se sobressaindo em relação aos demais. Este produto \'e a bermuda lisa com faturamento final de R$\$$ 36.581,00. 

\hspace{-0.8cm} Provavelmente, este \'e um produto novo lan\c cado apenas nesta loja, visto que nas outras quatro este tipo de bermuda n\~ao \'e comercializada. Visto isso, a solu\c c\~ao mais f\'acil para esta empresa aumentar as vendas, \'e: testar esta bermuda lisa nas outras quatro lojas, pois este produto esta sendo respons\'avel por uma porcentagem alta do faturamento.




\end{document}


